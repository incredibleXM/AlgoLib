\documentclass[a4paper]{article}

\usepackage{amsmath}
\usepackage{amssymb}
\usepackage{cmap}
\usepackage{geometry}
\usepackage{hyperref}
\usepackage{indentfirst}
\usepackage{xeCJK}
\usepackage{minted}

\geometry{margin=1in}

\setCJKmainfont[BoldFont={SimHei}]
{SimSun}
\setCJKmonofont{FangSong}

\newcommand{\cppcode}[1]{
    \inputminted[mathescape]{cpp}{#1}
}

\title{Template}
\author{Anonym}

\begin{document}

\maketitle

\tableofcontents

\clearpage



\section{计算几何}

\subsection{半平面交}

\cppcode{computation-geometry/HPI.cpp}

\subsection{求交点}

\subsubsection{线段交}

\cppcode{computation-geometry/int-of-seg.cpp}

\subsubsection{直线交}

\cppcode{computation-geometry/int-of-line.cpp}

\subsection{凸包}
\subsubsection{Graham}
\cppcode{computation-geometry/Graham.cpp}

\subsubsection{Andrew}
\cppcode{computation-geometry/Andrew.cpp}



\section{数据结构}

\subsection{KMP}

\cppcode{string/kmp.cpp}

\subsection{AC自动机}

\cppcode{string/ACautomaton(lrj).cpp}

\subsection{后缀三姐妹}

\subsubsection{后缀数组}

\cppcode{string/SuffixArray.cpp}

\subsubsection{后缀自动机}

\cppcode{string/suffix-automaton.cpp}

\subsection{最长回文串Manacher算法}

\texttt{palindrome[i]}是以$\frac{i}{2}$为对称中心的最长回文串长度

\cppcode{string/manacher.cpp}

\subsection{字符串hash}

\cppcode{string/hash.cpp}

\subsection{字典树}

\cppcode{string/Trie(lsrs).cpp}


\subsection{ST表}

\cppcode{data-structure/SparseTable.cpp}

\subsection{并查集}

\cppcode{data-structure/Union-Find.cpp}


\subsection{树状数组}

\cppcode{data-structure/BIT.cpp}

\subsection{莫队算法}

\cppcode{data-structure/Mo-algorithm.cpp}

\subsection{线段树}

\cppcode{data-structure/segment-tree(xiaodao).cpp}

\subsection{主席树}

\cppcode{data-structure/persistent-segment-tree.cpp}


\subsection{Splay}

\cppcode{data-structure/Splay(Menci).cpp}





\section{动态规划}

\subsection{石子归并}

\cppcode{dynamic-programming/Stone-Merge.cpp}

\subsection{最长上升子序列}

\cppcode{dynamic-programming/LIS.cpp}

\subsection{最长公共上升子序列}

\cppcode{dynamic-programming/LCIS.cpp}

\subsection{数位dp}

\cppcode{dynamic-programming/Digit-dp.cpp}

\subsection{多重背包}

\cppcode{dynamic-programming/MultiPack.cpp}





\section{图论}

\subsection{建图}

\cppcode{graph/Graph-Set.cpp}

\subsection{拓扑排序}

\cppcode{graph/TopoSort.cpp}

\subsection{Dijkastra}

\cppcode{graph/Dijkstra(pq).cpp}

\subsection{SPFA}

\cppcode{graph/spfa.cpp}

\subsection{Floyd}

\cppcode{graph/Floyd-Warshall.cpp}

\subsection{Prim}

\cppcode{graph/Prim.cpp}

\subsection{Kruskal}

\cppcode{graph/Kruskal.cpp}

\subsection{Boruvka}

\cppcode{graph/Boruvka.cpp}

\subsection{强连通分量缩点}

\cppcode{graph/scc.cpp}

\subsection{LCA}

\cppcode{graph/LCA.cpp}

\subsection{Tarjan}

\cppcode{graph/Tarjan.cpp}

\subsection{匈牙利算法}

\cppcode{graph/bipartite-matching.cpp}

\subsection{Hopcroft-Carp二分图匹配}

\cppcode{graph/Hopcroft-Carp.cpp}

\subsection{最大流}

\cppcode{graph/Dinic.cpp}

\subsection{最小费用流}

\cppcode{graph/min-cost-flow.cpp}






\section{数学}

\subsection{GCD/exGCD}

\cppcode{math/gcd.cpp}

\subsection{快速幂}

\cppcode{math/Fast-Power.cpp}

\subsection{乘法逆元}

\cppcode{math/multiplicative-inverse.cpp}

\subsection{模线性方程组}

\cppcode{math/Congruent-Equations.cpp}

\subsection{素数判断}

\cppcode{math/prime-judge.cpp}

\subsection{素数筛与素因子分解}

\cppcode{math/prime-sieve.cpp}

\subsection{区间筛}

\cppcode{math/segment-sieve.cpp}

\subsection{SG函数}

\cppcode{math/SG-function.cpp}

\subsection{欧拉函数}

\cppcode{math/euler-function.cpp}

\subsection{莫比乌斯函数}

\cppcode{math/mobius-function.cpp}

\subsection{莫比乌斯反演}

\cppcode{math/mobius-inversion.cpp}

\subsection{卢卡斯定理}

\cppcode{math/Lucas.cpp}

\subsection{高斯消元}

\cppcode{math/Gaussian-Elimination.cpp}

\subsection{线性基}

\cppcode{math/LinearBasis.cpp}

\subsection{格雷码}

\cppcode{math/gray-code.cpp}

\subsection{FFT}

\cppcode{math/FFT.cpp}

\subsection{NTT}

\cppcode{math/NTT.cpp}

\subsection{FWT}

\cppcode{math/FWT.cpp}


\subsection{自适应辛普森积分}

\cppcode{math/self-adapted-simpson.cpp}

\subsection{BM算法}

\cppcode{math/Berlekamp-Massey.cpp}

\subsection{millar-rabin}

\cppcode{millar-rabin.cpp}

\subsection{pollard-rho}

\cppcode{pollard-rho.cpp}

\subsection{PE Math}

\cppcode{math/PEmath.cpp}



\section{Miscellaneous}

\subsection{二分法}

\cppcode{tricks/binary-search.cpp}

\subsection{归并排序}

\cppcode{tricks/merge-sort.cpp}

\subsection{Fibonacci搜索}

\cppcode{tricks/Fibonacci-Search.cpp}

\subsection{位操作}

\cppcode{tricks/Bit-operation.cpp}

\subsection{平方根倒数}

\cppcode{tricks/Fast-Inverse-Square-Root.cpp}

\subsection{快速读写}

\cppcode{tricks/FastIO.cpp}

\subsection{直线下格点统计}

计算$$\sum_{0 \leq i < n} \lfloor \frac{a + b \cdot i}{m} \rfloor$$
($n, m > 0, a, b \geq 0$)
\cppcode{lattice-count.cpp}


\subsection{环状最长公共子串}

\cppcode{cyclic-longest-common-string.cpp}

\subsection{斯特林近似}

$$ n! = \sqrt{2 \pi n} (\frac{n}{e})^{n} $$
位数计算 $$ len = \frac{1}{2} log_{10}2 \pi n + nlog_{10}\frac{n}{e} + 1 $$

\subsection{二次剩余}

解方程$x^2 \equiv n \pmod p (p > 2)$, 
找$a$使得$\omega = a^2 - n$不是二次剩余,
则$$x \equiv (a + \sqrt{\omega})^{\frac{p + 1}{2}} \left(\bmod \frac{\mathbb{F}_p[x]}{(x^2 - \omega)}\right)$$

\subsection{五边形数定理}

设$p(n)$是$n$的拆分数,有$$p(n) = \sum_{k} (-1)^{k - 1} p\left(n - \frac{k(3k - 1)}{2}\right)$$

\subsection{球面三角公式}

设$a, b, c$是边长,$A, B, C$是所对的二面角,
有余弦定理$$\cos a = \cos b \cos c + \sin b \sin c \cos A$$
正弦定理$$\frac{\sin A}{\sin a} = \frac{\sin B}{\sin b} = \frac{\sin C}{\sin c}$$
三角形面积是$A + B + C - \pi$


\subsection{四面体体积公式}

$U, V, W, u, v, w$是四面体的$6$条棱,$U, V, W$构成三角形,$(U, u), (V, v), (W, w)$互为对棱,
则$$V = \frac{\sqrt{(s - 2a)(s - 2b)(s - 2c)(s - 2d)}}{192 uvw}$$
其中$$\left\{\begin{array}{lll}
        a & = & \sqrt{xYZ}, \\
        b & = & \sqrt{yZX}, \\
        c & = & \sqrt{zXY}, \\
        d & = & \sqrt{xyz}, \\
        s & = & a + b + c + d, \\ 
        X & = & (w - U + v)(U + v + w), \\
        x & = & (U - v + w)(v - w + U), \\
        Y & = & (u - V + w)(V + w + u), \\
        y & = & (V - w + u)(w - u + V), \\
        Z & = & (v - W + u)(W + u + v), \\
        z & = & (W - u + v)(u - v + W)
    \end{array}\right.$$

\subsection{牛顿恒等式}

设$$\prod_{i = 1}^n{(x - x_i)} = a_n + a_{n - 1} x + \dots + a_1 x^{n - 1} + a_0 x^n$$
$$p_k = \sum_{i = 1}^n{x_i^k}$$
则$$a_0 p_k + a_1 p_{k - 1} + \cdots + a_{k - 1} p_1 + k a_k = 0$$

特别地,对于$$|\mathbf{A} - \lambda \mathbf{E}| = (-1)^n(a_n + a_{n - 1} \lambda + \cdots + a_1 \lambda^{n - 1} + a_0 \lambda^n)$$
有$$p_k = \mathrm{Tr}(\mathbf{A}^k)$$

\end{document}
